\documentclass[11pt]{article}
\usepackage[utf8]{inputenc}
\usepackage[T1]{fontenc}
\usepackage{graphicx}
\usepackage{grffile}
\usepackage{longtable}
\usepackage{wrapfig}
\usepackage{rotating}
\usepackage[normalem]{ulem}
\usepackage{amsmath}
\usepackage{textcomp}
\usepackage{amssymb}
\usepackage{capt-of}
\usepackage{hyperref}
\usepackage{amssymb}
\author{Laurent Garnier}
\date{\today}
\title{Derivatives}
\hypersetup{
 pdfauthor={Laurent Garnier},
 pdftitle={Derivatives},
 pdfkeywords={},
 pdfsubject={},
 pdfcreator={Emacs 27.2 (Org mode 9.4)}, 
 pdflang={English}}
\begin{document}

\maketitle
\tableofcontents


\section{Source}
\label{sec:org3c243fb}

This content is a part of \href{https://laurentgarnier.podia.com/pratiques-mathematiques-en-1ere}{Pratiques Mathematiques} (in French)
created by Laurent Garnier.

This document (in English) has borrowed some inspiration from the
Oxford IB (International Baccalaureate) Diploma programme that you
can get on Amazon: \url{https://amzn.to/3SamRaK}

You can get the full course here:
\url{https://laurentgarnier.podia.com/pratiques-mathematiques-en-1ere}


\section{Basic Formulas}
\label{sec:org3c45c4c}
\subsection{General definition}
\label{sec:org72ab43d}

The derivative in mathematics is the rate at which one quantity
alters in relation to another. \textbf{Differentiation} is the name given to
the process of determining the derivative. The calculus is a branch
of mathematics that is based on these concepts.

\subsubsection{Example}
\label{sec:orgbb97393}

At any time \(t\) seconds, \(t > 1\), and \(d = -t^2 + t + 6\), a ball
traveling toward the edge of a ping-pong table is \(d\) cm away from
the edge. Calculate the ball's average speed between the first and
third seconds.

Average speed = total distance \(\div\) total time.

\[\dfrac{\left[-(3)^2 + 3 + 6\right]-\left[-(1)^2+1+6\right]}{3-1}
    = -3\]

The speed of the ball is 3 cm\(s^{-1}\).

\subsubsection{Definition 1}
\label{sec:orgd997582}

Typically, a function's average rate between two input values, \(x_1\)
and \(x_2\), is provided by

\[\dfrac{\Delta y}{\Delta x} = \dfrac{f(x_2)-f(x_1)}{x_2-x_1}\]

(read as 'the change in \(y\) divided by the change in \(x\)' where
\(\Delta\) is the Greek letter delta.)


\subsubsection{Definition 2}
\label{sec:orga7f3571}

The gradient of a curve \[y = f(x)\] at the point \[(x_0, f(x_0))\] is
\[ \lim_{h\to 0}\dfrac{f(x_0 + h) - f(x_0)}{h} \]  provided this
limit exists.

If this limit exists the gradient of of a curve \[y = f(x)\] at the
point \[(x_0, f(x_0))\] is written \[f'(x_0)\]

\subsubsection{Examples}
\label{sec:orgf7b4a1f}
\begin{enumerate}
\item Example 1
\label{sec:orgda5591e}

Find the gradient of the curve \(y = x^2\) at the point \(x_0 = -2\).

\item Solution 1
\label{sec:orgfc2d898}


\begin{align*}
\dfrac{\Delta y}{\Delta x} &= \dfrac{(-2+h)^2-(-2)^2}{(-2+h)-(-2)}\\
\dfrac{\Delta y}{\Delta x} &= \dfrac{4-4h+h^2-4}{h}\\
\dfrac{\Delta y}{\Delta x} &= \dfrac{-4h+h^2}{h}\\
\dfrac{\Delta y}{\Delta x} &= \dfrac{h(-4+h)}{h}\\
\dfrac{\Delta y}{\Delta x} &= -4+h
\end{align*}


Hence \[\lim_{h\to 0} (-4+h) = -4\]



\item Example 2
\label{sec:orge653c89}

Find the points on the curve \(y = \dfrac{1}{x}\) such that the
gradient at these points is \(-\dfrac{1}{9}\).

\item Solution 2
\label{sec:org881f60d}

Consider the point \(\left(x_0, \dfrac{1}{x_0}\right)\) and a
neighboring point \(\left(x_0 + h, \dfrac{1}{x_0 + h}\right)\).



\begin{align*}
\dfrac{\Delta y}{\Delta x} &= \dfrac{\dfrac{1}{x_0 + h}-\dfrac{1}{x_0}}{(x_0+h)-x_0}\\
\dfrac{\Delta y}{\Delta x} &= \dfrac{\dfrac{x_0-(x_0+h)}{x_0(x_0+h)}}{h}\\
\dfrac{\Delta y}{\Delta x} &= \dfrac{\dfrac{-h}{x_0^2+x_0h}}{h}\\
\dfrac{\Delta y}{\Delta x} &= \dfrac{-1}{x_0^2+x_0h}
\end{align*}


Therefore:

\[\lim_{h\to 0} \dfrac{-1}{x_0^2+x_0h} = -\dfrac{1}{x_0^2}\]

So \(-\dfrac{1}{x_0^2} = -\dfrac{1}{9}\) hence \(x_0 = \pm 3\).

The points are \(\left(3, \dfrac{1}{3}\right)\) and \(\left(-3,
     -\dfrac{1}{3}\right)\).
\end{enumerate}





\subsubsection{Exercices}
\label{sec:orgf53ca17}

\begin{enumerate}
\item Find the gradient of the curve at the given value of \(x\).
\begin{enumerate}
\item \(y = 3x^2 - 2x - 1\) at \(x = 5\)
\item \(y = \dfrac{3}{x}\) at \(x = -2\)
\item \(y = x^3\) at \(x = 7\)
\end{enumerate}
\item Find the point on the curve \(y = \dfrac{1}{x^2}\), such that the
gradient at the point is 3.
\item Find the gradient function of the curve \(y = 2x^2 +
       \dfrac{1}{x}\) and then the point on the curve where the
gradient is 3.
\end{enumerate}

\subsubsection{Definition 3}
\label{sec:org0c4a22d}

The \textbf{derivative}, or \textbf{gradient function}, of a function \(f\) with
respect to \(x\) is the function \[f'(x) = \lim_{h\to
    0}\dfrac{f(x+h)-f(x)}{h}\], provided this limit exists.

If \(f'\) exists, then \(f\) has a derivative at \(x\), or is
\textbf{differentiable} at \(x\). (\(f'(x)\) is read '\(f\) dash' or '\(f\)
prime', of \(x\).) Another notation for the derivative is
\(\dfrac{dy}{dx}\), the derivative of the function \(y = f(x)\) with
respect \(x\).

A function is differentiable if the derivative exists for all \(x\)
in the domain of \(f\).

\subsubsection{Examples}
\label{sec:orgf7267b0}
\begin{enumerate}
\item Example 1
\label{sec:orgb27dff6}

Find \(f'(x)\) given that \(f(x) = 2x^2 + x\), and hence find the
gradient of the function at \(x = -3\).

\begin{align*}
f'(x) &= \lim_{h\to 0}\dfrac{2(x+h)^2+(x+h)-(2x^2+x)}{h}\\
f'(x) &= \lim_{h\to 0}(4x+1+2h)\\
f'(x) &= 4x+1\\
f'(-3) &= -11
\end{align*}

\item Example 2
\label{sec:org8d96471}

A particle moves in a straight line so that its position from its
starting point at any time \(t\), in seconds, is given by \(s =
     4t^2\), where \(s\) is in metres. The particle passes through a
point \(P\) when \(t = a\) and then sometime later it passes through
point \(Q\) when \(t = a + h\). Find the average velocity as the
particle travels from point \(P\) to point \(Q\), and deduce its
velocity at the instant it passes through \(P\).

\(P(a, 4a^2)\) and \(Q(a+h, 4(a+h)^2)\)

\begin{align*}
\text{Average velocity } &= \dfrac{4(a+h)^2-4a^2}{(a+h)-a}\\
\text{Average velocity } &= \dfrac{4(a^2+2ah+h^2)-4a^2}{h}\\
\text{Average velocity } &= \dfrac{4h^2+8ah}{h}\\
\text{Average velocity } &= h\dfrac{4h+8a}{h}\\
\text{Average velocity } &= 4h+8
\end{align*}

Velocity at P = 8am\(s^{-1}\).
\end{enumerate}

\subsubsection{Exercices}
\label{sec:org5e89ccc}

\begin{enumerate}
\item Find the gradient function of the given curve, and then the
value of the gradient to the curve at the given point.
\begin{enumerate}
\item \(y = 4x^2 - 5x + 1\) at \(x = \dfrac{3}{8}\).
\item \(y = \sqrt{x}\) at \(x = 4\).
\item \(y = \dfrac{2}{x}\) at \(x = 3\).
\item \(y = \sqrt{x - 2}\) at \(x = 11\).
\item \(y = \dfrac{1}{\sqrt{x}}\) at \(x = 25\).
\end{enumerate}
\item A particle moves in a straight line so that its position from
its starting point after \(t\) seconds is \(12-5t^2\). If the
particles passes through point \(A\) when \(t = a\), and point \(B\)
when \(t = a + h\), find
\begin{enumerate}
\item the average velocity of the object as it moves from \(A\) to
\(B\)
\item the velocity as it passes through point \(A\).
\end{enumerate}
\end{enumerate}
\end{document}